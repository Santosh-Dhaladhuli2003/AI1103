\documentclass[journal,12pt,twocolumn]{IEEEtran}

\usepackage{setspace}
\usepackage{paralist}
\usepackage{gensymb}
\singlespacing
\usepackage[cmex10]{amsmath}

\usepackage{amsthm}
\usepackage{amssymb}

\usepackage{mathrsfs}
\usepackage{txfonts}
\usepackage{stfloats}
\usepackage{bm}
\usepackage{cite}
\usepackage{cases}
\usepackage{subfig}

\usepackage{longtable}
\usepackage{multirow}

\usepackage{enumitem}
\usepackage{mathtools}
\usepackage{steinmetz}
\usepackage{tikz}
\usepackage{circuitikz}
\usepackage{verbatim}
\usepackage{tfrupee}
\usepackage[breaklinks=true]{hyperref}
\usepackage{graphicx}
\usepackage{tkz-euclide}

\usetikzlibrary{calc,math}
\usepackage{listings}
    \usepackage{color}                                            %%
    \usepackage{array}                                            %%
    \usepackage{longtable}                                        %%
    \usepackage{calc}                                             %%
    \usepackage{multirow}                                         %%
    \usepackage{hhline}                                           %%
    \usepackage{ifthen}                                           %%
    \usepackage{lscape}     
\usepackage{multicol}
\usepackage{chngcntr}
\usepackage{mathtools}
\DeclarePairedDelimiter\ceil{\lceil}{\rceil}
\DeclarePairedDelimiter\floor{\lfloor}{\rfloor}

\DeclareMathOperator*{\Res}{Res}

\renewcommand\thesection{\arabic{section}}
\renewcommand\thesubsection{\thesection.\arabic{subsection}}
\renewcommand\thesubsubsection{\thesubsection.\arabic{subsubsection}}

\renewcommand\thesectiondis{\arabic{section}}
\renewcommand\thesubsectiondis{\thesectiondis.\arabic{subsection}}
\renewcommand\thesubsubsectiondis{\thesubsectiondis.\arabic{subsubsection}}


\hyphenation{op-tical net-works semi-conduc-tor}
\def\inputGnumericTable{}                                 %%

\lstset{
%language=C,
frame=single, 
breaklines=true,
columns=fullflexible
}
\begin{document}

\newcommand{\BEQA}{\begin{eqnarray}}
\newcommand{\EEQA}{\end{eqnarray}}
\newcommand{\define}{\stackrel{\triangle}{=}}
\bibliographystyle{IEEEtran}
\raggedbottom
\setlength{\parindent}{0pt}
\providecommand{\mbf}{\mathbf}
\providecommand{\pr}[1]{\ensuremath{\Pr\left(#1\right)}}
\providecommand{\qfunc}[1]{\ensuremath{Q\left(#1\right)}}
\providecommand{\sbrak}[1]{\ensuremath{{}\left[#1\right]}}
\providecommand{\lsbrak}[1]{\ensuremath{{}\left[#1\right.}}
\providecommand{\rsbrak}[1]{\ensuremath{{}\left.#1\right]}}
\providecommand{\brak}[1]{\ensuremath{\left(#1\right)}}
\providecommand{\lbrak}[1]{\ensuremath{\left(#1\right.}}
\providecommand{\rbrak}[1]{\ensuremath{\left.#1\right)}}
\providecommand{\cbrak}[1]{\ensuremath{\left\{#1\right\}}}
\providecommand{\lcbrak}[1]{\ensuremath{\left\{#1\right.}}
\providecommand{\rcbrak}[1]{\ensuremath{\left.#1\right\}}}
\theoremstyle{remark}
\newtheorem{rem}{Remark}
\newcommand{\sgn}{\mathop{\mathrm{sgn}}}
\providecommand{\abs}[1]{\vert#1\vert}
\providecommand{\res}[1]{\Res\displaylimits_{#1}} 
\providecommand{\norm}[1]{\lVert#1\rVert}
%\providecommand{\norm}[1]{\lVert#1\rVert}
\providecommand{\mtx}[1]{\mathbf{#1}}
\providecommand{\mean}[1]{E[ #1 ]}
\providecommand{\fourier}{\overset{\mathcal{F}}{ \rightleftharpoons}}
%\providecommand{\hilbert}{\overset{\mathcal{H}}{ \rightleftharpoons}}
\providecommand{\system}{\overset{\mathcal{H}}{ \longleftrightarrow}}
	%\newcommand{\solution}[2]{\textbf{Solution:}{#1}}
\newcommand{\solution}{\noindent \textbf{Solution: }}
\newcommand{\cosec}{\,\text{cosec}\,}
\providecommand{\dec}[2]{\ensuremath{\overset{#1}{\underset{#2}{\gtrless}}}}
\newcommand{\myvec}[1]{\ensuremath{\begin{pmatrix}#1\end{pmatrix}}}
\newcommand{\mydet}[1]{\ensuremath{\begin{vmatrix}#1\end{vmatrix}}}
\numberwithin{equation}{subsection}
\makeatletter
\@addtoreset{figure}{problem}
\makeatother
\let\StandardTheFigure\thefigure
\let\vec\mathbf
\renewcommand{\thefigure}{\theproblem}
\def\putbox#1#2#3{\makebox[0in][l]{\makebox[#1][l]{}\raisebox{\baselineskip}[0in][0in]{\raisebox{#2}[0in][0in]{#3}}}}
     \def\rightbox#1{\makebox[0in][r]{#1}}
     \def\centbox#1{\makebox[0in]{#1}}
     \def\topbox#1{\raisebox{-\baselineskip}[0in][0in]{#1}}
     \def\midbox#1{\raisebox{-0.5\baselineskip}[0in][0in]{#1}}
\vspace{3cm}
\title{\textbf{AI1103 : Assignment 3}}
\author{\textbf{Santosh Dhaladhuli MS20BTECH11007}}
\maketitle
\newpage
\bigskip
\renewcommand{\thefigure}{\theenumi}
\renewcommand{\thetable}{\theenumi}

Download all latex codes from 
\begin{lstlisting}
https://github.com/Santosh-Dhaladhuli2003/AI1103/blob/main/Assignment%203/Assignment%203.tex
\end{lstlisting}
\section{\textbf{GATE MA 2005 Question No. 25}}
Let $A_{1},A_{2},.....A_{n}$ be n independent events in which the Probability of occurence of the event $A_{i}$ is given by P($A_{i}$) = 1 - $\frac{1}{\alpha^i}$, $\alpha >1$, i = 1,2,3,....n.Then the probability that atleast one of the events occurs is

\begin{inparaenum}[(a)]
    \item  1 - $\frac{1}{\alpha^\frac{n(n+1)}{2}}$ \hspace{0.95cm}
    \item  $\frac{1}{\alpha^\frac{n(n+1)}{2}}$\hspace{1.5cm}
    \\ \\
    \item  $\frac{1}{\alpha^n}$ \hspace{2.15cm}
    \item 1 - $\frac{1}{\alpha^n}$\hspace{0.95cm}
  \end{inparaenum}

\section{\textbf{Solution}}
Let $A_{1} + A_{2} + A_{3} .... + A_{n}$ = S, \\
$\pr{S}$ = Probability of atleast one event occuring
De morgan's law states that $(A + B)^\prime = A^\prime B^\prime$  
\begin{align}
    \tag{1.1}
   \implies \pr{S} = 1 - \pr{S^\prime} \\ 
   \tag{1.2}
   1 - \pr{S^\prime}= 1 - \pr{A_{1}^\prime A_{2}^\prime A_{3}^\prime
   ....A_{n}^\prime}
   \label{1}
\end{align}
$\forall$ i $\in$ {1,2,....n} \\
Since, $A_{i}$ are independent.\\
$\therefore$ Complements of $A_{i}$ are also independent.\\
$\implies$ 
\begin{equation}
\tag{2.1}
\pr{A_{1}^\prime A_{2}^\prime A_{3}^\prime
   ....A_{n}^\prime}=\prod_{i=1}^{n}\pr{A_{i}^\prime}
   \label{2}
\end{equation}
\begin{equation}
    \tag{2.2}
\pr{A_{i}} = 1 - \frac{1}{\alpha^i} \implies \pr{A_{i}^\prime} = \frac{1}{\alpha^i} \label{5}
\end{equation}
substituting \eqref{5} in \eqref{2},
\begin{equation}
    \tag{2.3}
  \pr{A_{1}^\prime A_{2}^\prime A_{3}^\prime ....A_{n}^\prime} =  \prod_{i=1}^{n}\frac{1}{\alpha^i} \\
\end{equation}
\begin{equation}
\tag{2.4}
   \prod_{i=1}^{n}\frac{1}{\alpha^i}=\frac{1}{\alpha^{\sum_{i}^{n}i}}= \frac{1}{\alpha^\frac{n(n+1)}{2}} 
\end{equation}
\begin{equation}
\tag{2.5}
    \therefore \pr{A_{1}^\prime A_{2}^\prime A_{3}^\prime ....A_{n}^\prime} = \pr{S^\prime} = \frac{1}{\alpha^\frac{n(n+1)}{2}} \label{3}
\end{equation}
from equations \eqref{1} and \eqref{3} 
\begin{equation}
\tag{2.6}
\implies \pr{S} = 1 - \pr{S^\prime} = 1 - \frac{1}{\alpha^\frac{n(n+1)}{2}}
\end{equation}
$\therefore$ The correct option is \textbf{(a)}
\end{document}
