\documentclass[l1pt,a4paper,two column]{article}
\usepackage[utf8]{inputenc}

\title{Assignment-1}
\author{D.S.N. SANTOSH MS20BTECH11007}

\date{}
\usepackage{amsmath}
\usepackage{amssymb}
\usepackage{textcomp}








\begin{document}






\maketitle

Download all python codes from 
\\
\begin{lstlisting}
https://github.com/Santosh-Dhaladhuli2003/Assignment-1/blob/main/assigment_1.py
\end{lstlisting}
%
%
\\
\\
Download all latex-tikz codes from 
%
\\
\begin{lstlisting}
https://github.com/Santosh-Dhaladhuli2003/Assignment-1/blob/main/Assignment%201.tex
\end{lstlisting}
   



\section{QUESTION 5.12}
A Random Variable X has the following Probability Distribution \\
\begin{table}[h]
\begin{tabular}{|l|l|l|l|l|l|l|l|l|}

\hline
X    & 0 & 1 & 2  & 3  & 4  & 5   & 6  & 7                     \\ \hline    
P(X) & 0 & k & 2k & 2k & 3k & k^2 & 2k^2 & 7k^2 + k \\ 
\hline

\end{tabular}
\end{table}


\begin{enumerate}
    


Determine: \\
(i)   k \\
(ii)  P(X $<$ 3) \\
(iii) P(X $>$ 6)  \\
(iv)  P(0 $<$ X $<$ 3) \\
\\
\section{SOLUTION}
(i)
It is known that the sum of probabilities of a probability distribution is always one. \\


    

$
 \therefore 0 + k + 2k + 3k + k^2 + 2k^2 + (7k^2 + k) = 1 
 \\
 \implies 10k^2 + 9k - 1 = 0 \\
 \implies  (10k - 1)(k + 1) = 0 \\
 \implies  k = -1, \frac{1}{10} \\
  (K = -1) \\
  impossible
  $
  \\
  \begin{enumerate}
      \item \therefore k = \frac{1}{10}
  \end{enumerate}
 \begin{enumerate}
(ii)

    P(X$<$3) = P(X=2) + P(X=1) + P(X=0)

      \implies  P(X < 3) = 0 + k + 2k = 3k \\
      \therefore  P(X < 3) = \frac{3}{10} 
      \end{enumerate}
 \\
\begin{enumerate}
    \\
(iii)
P( X $>$ 6) = P(X = 7) = 7$k^2$ + k \\
\implies  P(X > 6)  = \frac{7}{100} + \frac{1}{10} \\
          \therefore P(X > 6) = \frac{17}{100} 
\end{enumerate}
\\
\begin{enumerate}

(iv)
 P(0 $<$ X $<$ 3) = P(X = 1) + P(X = 2)  \\
\implies   P(0 < X < 3) = k + 2k \\
\implies   P(0 < X < 3) = 3k \\
    \therefore    P(0 < X < 3) = \frac{3}{10} 
\end{enumerate}

\end{enumerate}



\end{document}
