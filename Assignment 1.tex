\documentclass[journal,12pt,twocolumn]{IEEEtran}

\usepackage{setspace}
\usepackage{gensymb}
\singlespacing
\usepackage[cmex10]{amsmath}
\usepackage{amssymb}
\usepackage{xurl}

\usepackage{amsthm}
\usepackage{comment}
\usepackage{mathrsfs}
\usepackage{txfonts}
\usepackage{stfloats}
\usepackage{bm}
\usepackage{cite}
\usepackage{cases}
\usepackage{subfig}

\usepackage{longtable}
\usepackage{multirow}

\usepackage{enumitem}
\usepackage{mathtools}
\usepackage{steinmetz}
\usepackage{tikz}
\usepackage{circuitikz}
\usepackage{verbatim}
\usepackage{tfrupee}
\usepackage[breaklinks=true]{hyperref}
\usepackage{graphicx}
\usepackage{tkz-euclide}

\usetikzlibrary{calc,math}
\usepackage{listings}
    \usepackage{color}                                            %%
    \usepackage{array}                                            %%
    \usepackage{longtable}                                        %%
    \usepackage{calc}                                             %%
    \usepackage{multirow}                                         %%
    \usepackage{hhline}                                           %%
    \usepackage{ifthen}                                           %%
    \usepackage{lscape}     
\usepackage{multicol}
\usepackage{chngcntr}

\DeclareMathOperator*{\Res}{Res}

\renewcommand\thesection{\arabic{section}}
\renewcommand\thesubsection{\thesection.\arabic{subsection}}
\renewcommand\thesubsubsection{\thesubsection.\arabic{subsubsection}}

\renewcommand\thesectiondis{\arabic{section}}
\renewcommand\thesubsectiondis{\thesectiondis.\arabic{subsection}}
\renewcommand\thesubsubsectiondis{\thesubsectiondis.\arabic{subsubsection}}


\hyphenation{op-tical net-works semi-conduc-tor}
\def\inputGnumericTable{}                                 %%

\lstset{
%language=C,
frame=single, 
breaklines=true,
columns=fullflexible
}
\begin{document}


\newtheorem{theorem}{Theorem}[section]
\newtheorem{problem}{Problem}
\newtheorem{proposition}{Proposition}[section]
\newtheorem{lemma}{Lemma}[section]
\newtheorem{corollary}[theorem]{Corollary}
\newtheorem{example}{Example}[section]
\newtheorem{definition}[problem]{Definition}

\newcommand{\BEQA}{\begin{eqnarray}}
\newcommand{\EEQA}{\end{eqnarray}}
\newcommand{\define}{\stackrel{\triangle}{=}}
\bibliographystyle{IEEEtran}
\raggedbottom
\setlength{\parindent}{0pt}
\providecommand{\mbf}{\mathbf}
\providecommand{\pr}[1]{\ensuremath{\Pr\left(#1\right)}}
\providecommand{\qfunc}[1]{\ensuremath{Q\left(#1\right)}}
\providecommand{\sbrak}[1]{\ensuremath{{}\left[#1\right]}}
\providecommand{\lsbrak}[1]{\ensuremath{{}\left[#1\right.}}
\providecommand{\rsbrak}[1]{\ensuremath{{}\left.#1\right]}}
\providecommand{\brak}[1]{\ensuremath{\left(#1\right)}}
\providecommand{\lbrak}[1]{\ensuremath{\left(#1\right.}}
\providecommand{\rbrak}[1]{\ensuremath{\left.#1\right)}}
\providecommand{\cbrak}[1]{\ensuremath{\left\{#1\right\}}}
\providecommand{\lcbrak}[1]{\ensuremath{\left\{#1\right.}}
\providecommand{\rcbrak}[1]{\ensuremath{\left.#1\right\}}}
\theoremstyle{remark}
\newtheorem{rem}{Remark}
\newcommand{\sgn}{\mathop{\mathrm{sgn}}}
\providecommand{\abs}[1]{\vert#1\vert}
\providecommand{\res}[1]{\Res\displaylimits_{#1}} 
\providecommand{\norm}[1]{\lVert#1\rVert}
%\providecommand{\norm}[1]{\lVert#1\rVert}
\providecommand{\mtx}[1]{\mathbf{#1}}
\providecommand{\mean}[1]{E[ #1 ]}
\providecommand{\fourier}{\overset{\mathcal{F}}{ \rightleftharpoons}}
%\providecommand{\hilbert}{\overset{\mathcal{H}}{ \rightleftharpoons}}
\providecommand{\system}{\overset{\mathcal{H}}{ \longleftrightarrow}}
	%\newcommand{\solution}[2]{\textbf{Solution:}{#1}}
\newcommand{\solution}{\noindent \textbf{Solution: }}
\newcommand{\cosec}{\,\text{cosec}\,}
\providecommand{\dec}[2]{\ensuremath{\overset{#1}{\underset{#2}{\gtrless}}}}
\newcommand{\myvec}[1]{\ensuremath{\begin{pmatrix}#1\end{pmatrix}}}
\newcommand{\mydet}[1]{\ensuremath{\begin{vmatrix}#1\end{vmatrix}}}
\numberwithin{equation}{subsection}
\makeatletter
\@addtoreset{figure}{problem}
\makeatother
\let\StandardTheFigure\thefigure
\let\vec\mathbf
\renewcommand{\thefigure}{\theproblem}
\def\putbox#1#2#3{\makebox[0in][l]{\makebox[#1][l]{}\raisebox{\baselineskip}[0in][0in]{\raisebox{#2}[0in][0in]{#3}}}}
     \def\rightbox#1{\makebox[0in][r]{#1}}
     \def\centbox#1{\makebox[0in]{#1}}
     \def\topbox#1{\raisebox{-\baselineskip}[0in][0in]{#1}}
     \def\midbox#1{\raisebox{-0.5\baselineskip}[0in][0in]{#1}}
\vspace{3cm}
\title{AI1103 : Assignment 1}
\author{Santosh Dhaladhuli - MS20BTECH11007}
\maketitle
\newpage
\bigskip
\renewcommand{\thefigure}{\theenumi}
\renewcommand{\thetable}{\theenumi}
Download all python codes from 
\begin{lstlisting}
https://github.com/Santosh-Dhaladhuli2003/AI1103-Assignment-1/blob/main/Assignment%201.py
\end{lstlisting}
%
and latex codes from 
%
\begin{lstlisting}
https://github.com/Santosh-Dhaladhuli2003/AI1103-Assignment-1/blob/main/Assignment%201.tex
\end{lstlisting}
\section*{PROBLEM 5.12}
Random Variable X has the following Probability Distribution \\
\begin{table}[h]
\begin{tabular}{|l|l|l|l|l|l|l|l|l|}
\hline
X    & 0 & 1 & 2  & 3  & 4  & 5   & 6  & 7                     \\ \hline    
P(X) & 0 & k & 2k & 2k & 3k & $k^2$ & $2k^2$ & $7k^2$ + k \\ 
\hline
\end{tabular}
\end{table} 
\\
Determine:
\begin{enumerate}
\item k 
\item \pr{X < 3} 
\item \pr{X > 6}  
\item \pr{0 < X < 3}
\end{enumerate}
\section*{Solution}

PMF of X:
\begin{align}
 \tag{5.12.1}
 \pr{X} = 
  \begin{cases}
    0, & \text{for }  X = 0 \\
    k, & \text{for }  X = 1 \\
    2k, & \text{for } X = 2 \\
    2k, & \text{for } X = 3 \\
    3k, & \text{for } X = 4 \\
    k^2, & \text{for } X = 5 \\
    2k^2, & \text{for } X = 6 \\
    7k^2 + k, & \text{for } X = 7 \\
    \end{cases}
  \end{align}
\begin{enumerate}
    \item It is known that the sum of probabilities of a probability distribution is always one. 
\begin{align}
\tag{5.12.2}
    \therefore 0 + k + 2k + 3k + k^2 + 2k^2 + (7k^2 + k) = 1 
\end{align}
\begin{align}
\tag{5.12.3}
\implies 10k^2 + 9k - 1 = 0 
 \implies  (10k - 1)(k + 1) = 0 
 \end{align}
  \begin{equation}
  \tag{5.12.4}
 \implies  k = -1, \frac{1}{10} 
\end{equation}
\begin{equation}
    \tag{1}
 \therefore k = \frac{1}{10} (\because k \ge 0)
 \end{equation}
 
 \begin{figure}[!htb]
    \centering    
	\includegraphics[width=\columnwidth]{probability.png}
    \caption{Probability Mass Function(PMF)}
\end{figure}
 
 CDF of X:
 \begin{table}[h]
\begin{tabular}{|l|l|l|l|l|l|l|l|l|}
\hline
X    & 0 & 1    & 2    & 3    & 4    & 5      & 6      & 7 \\ \hline
F(X) & 0 & 1/10 & 3/10 & 5/10 & 8/10 & 81/100 & 83/100 & 1 \\ \hline
\end{tabular}
\end{table}
\begin{figure}[!htb]
    \centering    
	\includegraphics[width=\columnwidth]{CDF.png}
    \caption{Cummulative Distribution Function(CDF)}
\end{figure}
\\
 We know that \pr{X \le x} = F(x) \\
 and  \pr{x< X \le y} = F(y) - F(x) \\
\item \pr{X < 3} = \pr{X \le 3} - \pr{X = 3} 
 \begin{align}
 \tag{5.12.5}
 \implies \pr{X < 3} =F(3)- \pr{X = 3} \\
\tag{5.12.6}
\implies  \pr{X < 3} = \frac{5}{10} - \frac{2}{10}   \\
\tag{2}
\therefore  \pr{X < 3} = \frac{3}{10} 
   \end{align} 
 \item \pr{X > 6} = 1 - \pr{X \le 6} = 1 - F(6)
 \begin{align}
\tag{5.12.7}
\implies  \pr{X > 6}  = 1 - \frac{83}{100} \\
\tag{3}
\therefore \pr{X > 6} = \frac{17}{100} 
\end{align}
\item \pr{0 < X < 3} = \pr{0 < X \le 3} - \pr{X = 3}  
\begin{align}
\tag{5.12.8}
\implies \pr{0 < X < 3} = F(3) - F(0) - \pr{X = 3} \\
 \tag{5.12.9}
\implies   \pr{0 < X < 3} = \frac{5}{10} - 0 - \frac{2}{10} \\
\tag{4}              
    \therefore    \pr{0 < X < 3} = \frac{3}{10}
\end{align}

\end{enumerate}
\end{document}
