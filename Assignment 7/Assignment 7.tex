\documentclass[journal,12pt,twocolumn]{IEEEtran}

\usepackage{setspace}
\usepackage{paralist}
\usepackage{gensymb}
\singlespacing
\usepackage[cmex10]{amsmath}

\usepackage{amsthm}
\usepackage{amssymb}

\usepackage{mathrsfs}
\usepackage{txfonts}
\usepackage{stfloats}
\usepackage{bm}
\usepackage{cite}
\usepackage{cases}
\usepackage{subfig}

\usepackage{longtable}
\usepackage{multirow}

\usepackage{enumitem}
\usepackage{mathtools}
\usepackage{steinmetz}
\usepackage{tikz}
\usepackage{circuitikz}
\usepackage{verbatim}
\usepackage{tfrupee}
\usepackage[breaklinks=true]{hyperref}
\usepackage{graphicx}
\usepackage{tkz-euclide}

\usetikzlibrary{calc,math}
\usepackage{listings}
    \usepackage{color}                                            %%
    \usepackage{array}                                            %%
    \usepackage{longtable}                                        %%
    \usepackage{calc}                                             %%
    \usepackage{multirow}                                         %%
    \usepackage{hhline}                                           %%
    \usepackage{ifthen}                                           %%
    \usepackage{lscape}     
\usepackage{multicol}
\usepackage{chngcntr}
\usepackage{mathtools}
\DeclarePairedDelimiter\ceil{\lceil}{\rceil}
\DeclarePairedDelimiter\floor{\lfloor}{\rfloor}

\DeclareMathOperator*{\Res}{Res}

\renewcommand\thesection{\arabic{section}}
\renewcommand\thesubsection{\thesection.\arabic{subsection}}
\renewcommand\thesubsubsection{\thesubsection.\arabic{subsubsection}}

\renewcommand\thesectiondis{\arabic{section}}
\renewcommand\thesubsectiondis{\thesectiondis.\arabic{subsection}}
\renewcommand\thesubsubsectiondis{\thesubsectiondis.\arabic{subsubsection}}


\hyphenation{op-tical net-works semi-conduc-tor}
\def\inputGnumericTable{}                                 %%

\lstset{
%language=C,
frame=single, 
breaklines=true,
columns=fullflexible
}
\begin{document}

\newcommand{\BEQA}{\begin{eqnarray}}
\newcommand{\EEQA}{\end{eqnarray}}
\newcommand{\define}{\stackrel{\triangle}{=}}
\bibliographystyle{IEEEtran}
\raggedbottom
\setlength{\parindent}{0pt}
\providecommand{\mbf}{\mathbf}
\providecommand{\pr}[1]{\ensuremath{\Pr\left(#1\right)}}
\providecommand{\qfunc}[1]{\ensuremath{Q\left(#1\right)}}
\providecommand{\sbrak}[1]{\ensuremath{{}\left[#1\right]}}
\providecommand{\lsbrak}[1]{\ensuremath{{}\left[#1\right.}}
\providecommand{\rsbrak}[1]{\ensuremath{{}\left.#1\right]}}
\providecommand{\brak}[1]{\ensuremath{\left(#1\right)}}
\providecommand{\lbrak}[1]{\ensuremath{\left(#1\right.}}
\providecommand{\rbrak}[1]{\ensuremath{\left.#1\right)}}
\providecommand{\cbrak}[1]{\ensuremath{\left\{#1\right\}}}
\providecommand{\lcbrak}[1]{\ensuremath{\left\{#1\right.}}
\providecommand{\rcbrak}[1]{\ensuremath{\left.#1\right\}}}
\theoremstyle{remark}
\newtheorem{rem}{Remark}
\newcommand{\sgn}{\mathop{\mathrm{sgn}}}
\providecommand{\abs}[1]{\vert#1\vert}
\providecommand{\res}[1]{\Res\displaylimits_{#1}} 
\providecommand{\norm}[1]{\lVert#1\rVert}
%\providecommand{\norm}[1]{\lVert#1\rVert}
\providecommand{\mtx}[1]{\mathbf{#1}}
\providecommand{\mean}[1]{E[ #1 ]}
\providecommand{\fourier}{\overset{\mathcal{F}}{ \rightleftharpoons}}
%\providecommand{\hilbert}{\overset{\mathcal{H}}{ \rightleftharpoons}}
\providecommand{\system}{\overset{\mathcal{H}}{ \longleftrightarrow}}
	%\newcommand{\solution}[2]{\textbf{Solution:}{#1}}
\newcommand{\solution}{\noindent \textbf{Solution: }}
\newcommand{\cosec}{\,\text{cosec}\,}
\providecommand{\dec}[2]{\ensuremath{\overset{#1}{\underset{#2}{\gtrless}}}}
\newcommand{\myvec}[1]{\ensuremath{\begin{pmatrix}#1\end{pmatrix}}}
\newcommand{\mydet}[1]{\ensuremath{\begin{vmatrix}#1\end{vmatrix}}}
\numberwithin{equation}{subsection}
\makeatletter
\@addtoreset{figure}{problem}
\makeatother
\let\StandardTheFigure\thefigure
\let\vec\mathbf
\renewcommand{\thefigure}{\theproblem}
\def\putbox#1#2#3{\makebox[0in][l]{\makebox[#1][l]{}\raisebox{\baselineskip}[0in][0in]{\raisebox{#2}[0in][0in]{#3}}}}
     \def\rightbox#1{\makebox[0in][r]{#1}}
     \def\centbox#1{\makebox[0in]{#1}}
     \def\topbox#1{\raisebox{-\baselineskip}[0in][0in]{#1}}
     \def\midbox#1{\raisebox{-0.5\baselineskip}[0in][0in]{#1}}
\vspace{3cm}
\title{\textbf{AI1103 : Assignment 7}}
\author{\textbf{Santosh Dhaladhuli MS20BTECH11007}}
\maketitle
\newpage
\bigskip
\renewcommand{\thefigure}{\theenumi}
\renewcommand{\thetable}{\theenumi}

Download all latex codes from 
\begin{lstlisting}
https://github.com/Santosh-Dhaladhuli2003/AI1103/blob/main/Assignment%207/Assignment%207.tex
\end{lstlisting}
\section{\textbf{CSIR - UGC 2014 Dec Q.103}}
Suppose X is a Random Variable such that E(X) = 0, E($X^2$) = 2 and E($X^4$)=4.Then

\begin{enumerate}
\item  E($X^3$)=0
\item \pr{X \ge 0}= $\frac{1}{2}$
\item X $\sim$ N(0,2) 
\item \text{X is bounded with Probability} 1.
\end{enumerate}

\section{\textbf{Solution}}
Let X be a Random variable. \\
Compute Variance of $X^2$ 
\begin{align*}
    Var(X^2) &= E(X^4) - (E(X^2))^{2} \\
             &=  4 - 2^2 \\
             &= 0\\
\tag{1} \label{Var(X^2)}   \implies Var(X^2) &= 0
\end{align*}
$\therefore$ X is a random variable such that $X^2$ is constant.\\
\centering
Given E($X^2$) = 2,
\begin{align*}
    E(X^2) &= \Sigma X^2\pr{X} \\
           &= X^2\Sigma \pr{X}  \\
           &= X^2 (\because \Sigma \pr{X} = 1) \\
       X^2 &= 2 \\
\tag{2} \label{X}
\implies X &= \pm \sqrt{2}
\end{align*}
\centering
Given E(X) = 0,
\begin{align*}
                      E(X) = \Sigma X\pr{X} = 0 \\
                    \sqrt{2}\pr{X = \sqrt{2}}-\sqrt{2}\pr{X = -\sqrt{2}} = 0\\
      \tag{3} \label{P(X)}
\implies \pr{X = \sqrt{2}} = \pr{X = -\sqrt{2}}
\end{align*}
\centering
Also, Sum of Probabilities is 1,
\begin{align*}
    \implies \pr{X = \sqrt{2}} + \pr{X =-\sqrt{2}} &= 1 \\
    \tag{4} \label{X > 0}
    \implies                     \pr{X = \sqrt{2}} &= \frac{1}{2} \\
    \tag{5} \label{X < 0}
    \implies                    \pr{X = -\sqrt{2}} &= \frac{1}{2}
\end{align*}
\centering
\textbf{Option 1} says E($X^3$) = 0,
\begin{align*}
    E(X^3) &= \Sigma X^3 \pr{X} \\
           &= X^2 .\Sigma X \pr{X} \\
           &= X^2 E(X)\\
\implies E(X^3) &= 0      
\end{align*} 

\centering \boxed{\text{\textbf{Option 1} is a \textbf{correct} answer}}
\vspace{1cm} \\
\textbf{Option 2} says $\pr{X \ge 0}$ = $\frac{1}{2}$,
\begin{align*}
    \pr{X\ge 0} &= \pr{X = \sqrt{2}} = \frac{1}{2} \\
    \implies \pr{X \ge 0} &= \frac{1}{2}
\end{align*}

\centering \boxed{\text{\textbf{Option 2} is a \textbf{correct} answer}}
\vspace{1cm} \\
\textbf{Option 3} says X $\sim$ N(0,2),\\
Let $\mu$ be the mean of X
\begin{align*}
            \mu &= E(X) \\
    \tag{6} \label{mu}
   \implies \mu &= 0 \\
      \sigma^2 = Var(X) &= E(X^2) - (E(X))^{2} \\
                &= 2 - (0)^{2} \\
                \tag{7} \label{Var(X)}
\implies Var(X) &= 2
\implies N(\mu,\sigma^2) = N(0,2) \\
\implies X &\sim N(0,2)
\end{align*}
\centering \boxed{\text{\textbf{Option 3} is a \textbf{correct} answer}}
\pagebreak  \\
\centering
\textbf{Option 4} says X is bounded with probability 1, \\
Equations \eqref{X > 0} and \eqref{X < 0} show that X $\in ({-\sqrt{2},\sqrt{2}})$ with Probability 1. 
\vspace{0.5cm} \\
\centering \boxed{\text{\textbf{Option 4} is a \textbf{correct} answer}}
\end{document}
