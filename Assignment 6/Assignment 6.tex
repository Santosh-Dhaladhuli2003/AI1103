\documentclass[journal,12pt,twocolumn]{IEEEtran}

\usepackage{setspace}
\usepackage{paralist}
\usepackage{gensymb}
\singlespacing
\usepackage[cmex10]{amsmath}

\usepackage{amsthm}
\usepackage{amssymb}

\usepackage{mathrsfs}
\usepackage{txfonts}
\usepackage{stfloats}
\usepackage{bm}
\usepackage{cite}
\usepackage{cases}
\usepackage{subfig}

\usepackage{longtable}
\usepackage{multirow}

\usepackage{enumitem}
\usepackage{mathtools}
\usepackage{steinmetz}
\usepackage{tikz}
\usepackage{circuitikz}
\usepackage{verbatim}
\usepackage{tfrupee}
\usepackage[breaklinks=true]{hyperref}
\usepackage{graphicx}
\usepackage{tkz-euclide}

\usetikzlibrary{calc,math}
\usepackage{listings}
    \usepackage{color}                                            %%
    \usepackage{array}                                            %%
    \usepackage{longtable}                                        %%
    \usepackage{calc}                                             %%
    \usepackage{multirow}                                         %%
    \usepackage{hhline}                                           %%
    \usepackage{ifthen}                                           %%
    \usepackage{lscape}     
\usepackage{multicol}
\usepackage{chngcntr}
\usepackage{mathtools}
\DeclarePairedDelimiter\ceil{\lceil}{\rceil}
\DeclarePairedDelimiter\floor{\lfloor}{\rfloor}

\DeclareMathOperator*{\Res}{Res}

\renewcommand\thesection{\arabic{section}}
\renewcommand\thesubsection{\thesection.\arabic{subsection}}
\renewcommand\thesubsubsection{\thesubsection.\arabic{subsubsection}}

\renewcommand\thesectiondis{\arabic{section}}
\renewcommand\thesubsectiondis{\thesectiondis.\arabic{subsection}}
\renewcommand\thesubsubsectiondis{\thesubsectiondis.\arabic{subsubsection}}


\hyphenation{op-tical net-works semi-conduc-tor}
\def\inputGnumericTable{}                                 %%

\lstset{
%language=C,
frame=single, 
breaklines=true,
columns=fullflexible
}
\begin{document}

\newcommand{\BEQA}{\begin{eqnarray}}
\newcommand{\EEQA}{\end{eqnarray}}
\newcommand{\define}{\stackrel{\triangle}{=}}
\bibliographystyle{IEEEtran}
\raggedbottom
\setlength{\parindent}{0pt}
\providecommand{\mbf}{\mathbf}
\providecommand{\pr}[1]{\ensuremath{\Pr\left(#1\right)}}
\providecommand{\qfunc}[1]{\ensuremath{Q\left(#1\right)}}
\providecommand{\sbrak}[1]{\ensuremath{{}\left[#1\right]}}
\providecommand{\lsbrak}[1]{\ensuremath{{}\left[#1\right.}}
\providecommand{\rsbrak}[1]{\ensuremath{{}\left.#1\right]}}
\providecommand{\brak}[1]{\ensuremath{\left(#1\right)}}
\providecommand{\lbrak}[1]{\ensuremath{\left(#1\right.}}
\providecommand{\rbrak}[1]{\ensuremath{\left.#1\right)}}
\providecommand{\cbrak}[1]{\ensuremath{\left\{#1\right\}}}
\providecommand{\lcbrak}[1]{\ensuremath{\left\{#1\right.}}
\providecommand{\rcbrak}[1]{\ensuremath{\left.#1\right\}}}
\theoremstyle{remark}
\newtheorem{rem}{Remark}
\newcommand{\sgn}{\mathop{\mathrm{sgn}}}
\providecommand{\abs}[1]{\vert#1\vert}
\providecommand{\res}[1]{\Res\displaylimits_{#1}} 
\providecommand{\norm}[1]{\lVert#1\rVert}
%\providecommand{\norm}[1]{\lVert#1\rVert}
\providecommand{\mtx}[1]{\mathbf{#1}}
\providecommand{\mean}[1]{E[ #1 ]}
\providecommand{\fourier}{\overset{\mathcal{F}}{ \rightleftharpoons}}
%\providecommand{\hilbert}{\overset{\mathcal{H}}{ \rightleftharpoons}}
\providecommand{\system}{\overset{\mathcal{H}}{ \longleftrightarrow}}
	%\newcommand{\solution}[2]{\textbf{Solution:}{#1}}
\newcommand{\solution}{\noindent \textbf{Solution: }}
\newcommand{\cosec}{\,\text{cosec}\,}
\providecommand{\dec}[2]{\ensuremath{\overset{#1}{\underset{#2}{\gtrless}}}}
\newcommand{\myvec}[1]{\ensuremath{\begin{pmatrix}#1\end{pmatrix}}}
\newcommand{\mydet}[1]{\ensuremath{\begin{vmatrix}#1\end{vmatrix}}}
\numberwithin{equation}{subsection}
\makeatletter
\@addtoreset{figure}{problem}
\makeatother
\let\StandardTheFigure\thefigure
\let\vec\mathbf
\renewcommand{\thefigure}{\theproblem}
\def\putbox#1#2#3{\makebox[0in][l]{\makebox[#1][l]{}\raisebox{\baselineskip}[0in][0in]{\raisebox{#2}[0in][0in]{#3}}}}
     \def\rightbox#1{\makebox[0in][r]{#1}}
     \def\centbox#1{\makebox[0in]{#1}}
     \def\topbox#1{\raisebox{-\baselineskip}[0in][0in]{#1}}
     \def\midbox#1{\raisebox{-0.5\baselineskip}[0in][0in]{#1}}
\vspace{3cm}
\title{\textbf{AI1103 : Assignment 6}}
\author{\textbf{Santosh Dhaladhuli MS20BTECH11007}}
\maketitle
\newpage
\bigskip
\renewcommand{\thefigure}{\theenumi}
\renewcommand{\thetable}{\theenumi}

Download all latex codes from 
\begin{lstlisting}
https://github.com/Santosh-Dhaladhuli2003/AI1103/blob/main/Assignment%206/Assignment%206.tex
\end{lstlisting}
\section{\textbf{GATE ST 2021 Q.1 st. section}}
Let X be a non-constant positive Random Variable such that $E(X) = 9$.\\
Then which of the following statements is True?

\begin{enumerate}
\item  $E\brak{\frac{1}{X+1}} > 0.1$ and $\pr{X \ge 10} \le 0.9$
\item   $E\brak{\frac{1}{X+1}} < 0.1$ and $\pr{X \ge 10} \le 0.9$
\item   $E\brak{\frac{1}{X+1}} > 0.1$ and $\pr{X \ge 10} > 0.9$
\item   $E\brak{\frac{1}{X+1}} < 0.1$ and $\pr{X \ge 10} > 0.9$
\end{enumerate}

\section{\textbf{Solution}}
Given, for X $>$ 0 ,$E(X) = 9$, $E\brak{\frac{1}{X+1}}$ can be estimated by Jensens's Inequality. \\
\textbf{pre - requisites:}\\
In general, $\phi(X)$ is a convex function iff:
\begin{equation*}
    \frac{d^2 \phi}{dX^2} \ge 0
\end{equation*}
\textbf{Jensen's Inequality:}\\
In the context of probability theory, it is generally stated in the following form: if X is a random variable and $\phi$ is a convex function, then
\begin{equation*}
\tag{1} \label{jenson}
    \phi(E(X)) \le E(\phi(X))
\end{equation*}
\begin{align*}
    \text{So for } \phi(X) &= \frac{1}{X+1}, \\
                \frac{d\phi}{dX} &= - \frac{1}{(X+1)^{2}} \\
                \tag{2} \label{phi}
                \frac{d^2 \phi}{dX^2} &= \frac{2}{(X+1)^{3}} 
    \implies \frac{d^2 \phi}{dX^2} \ge 0,(\because X>0 )
\end{align*}
by eq \eqref{jenson} and \eqref{phi}
\begin{align*}
    E\brak{\frac{1}{X+1}} &\ge \frac{1}{E(X)+1} \\
    \implies E\brak{\frac{1}{X+1}} &\ge \frac{1}{9 + 1} \\
    \tag{3} \label{Part 1}
    \implies E\brak{\frac{1}{X+1}} &\ge 0.1
\end{align*}
$\pr{X \ge 10}$ can be estimated by Markov's Inequality.\\
\textbf{Markov's Inequality:}
If X is a non-negative random variable and a $>$ 0, then the probability that X is at least a is at most the expectation of X divided by a. \\
Mathematically,
\begin{equation*}
    \tag{4} \label{markov}
    \pr{X \ge a} \le \frac{E(X)}{a}
\end{equation*}
by \eqref{markov} for a = 10
\begin{align*}
    \pr{X \ge 10} &\le \frac{E(X)}{10} \\
    \implies \pr{X \ge 10} &\le \frac{9}{10} \\
    \tag{5} \label{Part 2}
   \therefore \pr{X \ge 10} &\le 0.9
\end{align*}
So, from \eqref{Part 1} and \eqref{Part 2} \\
\begin{center}
     \boxed{\textbf{Option 1 is the Correct Answer}}
\end{center}
\end{document}
